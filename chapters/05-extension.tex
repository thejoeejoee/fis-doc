\chapter{Rozšíření pro Node-RED}
\label{ch:rozsireni}

Na základně poznatků z~kapitoly~\ref{sec:node-red-rozsireni} a navrženého komunikačního protokolu zde navrhnu a
zrealizuji vlastní rozšíření nástroje Node-RED -- předmětem budou detaily ze základní infrastruktury bloků a
následně několik konkrétních
aplikací nasaditelných přímo v~reálném provozu.
% \marginpar{\small Poznámky na okraji}

\section{Základní konfigurační blok}\label{sec:zakladni-konfiguracni-blok}
Základním stavebním prvkem pro další bloky je \ic{fis-node} (reprezentovaný třídou \ic{FisNode}) -- využije se tak
konfiguračních bloků, díky kterým nebude nutné přihlašovací údaje k MQTT brokeru nebo identifikacím jednotlivých
uzlů zadávat více než jednou.

Nedělitelnou součástí každého bloku v Node-RED je jeho konfigurační formulář v editoru.
Ten se skládá ze dvou povinných částí a jedné nepovinné -- z povinných částí jedná se o programovou definici pro editor
sítě v jazyce Javascript a o samotnou definici konfiguračního formuláře popsanou v jazyce HTML.
Nepovinnou částí je poté uživatelská dokumentace dostupná přímo z editoru.
Zkracenou definici bloku pro editor lze vidět v ukázce~\ref{code:fis-node-editor}, která kromě samotné registrace
obsahuje dva důležité aspekty.

Prvním je registrace do kategorie \ic{\'config\'} což značí registraci konfigurační bloku pro reprezentaci jednoho IoT
uzlu (uzel tedy nemá grafickou reprezentaci v síti).
Druhým aspektem je výčet parametrů pro formulář, které následně bude možné použít pro samotné chování uzlu v síti.
První parametr typu \ic{\'mqtt-broker\'} je určen pro konkrétní MQTT broker, pomocí kterého bude blok s uzlem spojen
(jedná se opět o konfigurační blok poskytnutý přímo nástrojem Node-RED).

Druhým parametrem je jednoznačná identifikace bloku, jejíž formát je omezen regulárním výrazem -- tento
parametr vychází z navrženého protokolu popsaného v kapitole~\ref{sec:mqtt-kanaly}\todo{čárka před "a"?} a to
konkrétně ze zástupného symbolu \ic{NODE\_ID}, který od sebe odlišuje jednotlivé uzly\footnote{Konkrétní tvar
identifikátoru (a jeho regulárního výrazu) není pro Node-RED podstatný, k jeho upřesnění dojde v rámci popisu firmwaru
uzlů v kapitole~\ref{ch:firmware}.}.
Parametr \ic{nodeId} tedy následně bude použit pro sestavení MQTT kanálů pro odběr a publikaci zpráv.

% @formatter:off
\begin{code}[
    language=Javascript,
    label=code:fis-node-editor,
    caption={Registrace vlastního bloku do editoru sítě v nástroji Node-RED.}
]
RED.nodes.registerType('fis-node', {
    category: 'config',
    defaults: {
        broker: {type: "mqtt-broker", required: true, value: ""},
        nodeId: {
            value: "", required: true,
            validate: RED.validators.regex(/^[a-z0-9]{12}(*\textdollar*)/i)
        },
    },
    (*\ldots*)
});
\end{code}

V ukázce~\ref{code:fis-node-constructor} je zobrazena část třídy \ic{FisNode}, reprezentující blok z hlediska sítě. V
konstruktoru této třídy dochází k získání konfiguračních
parametrů z formuláře, získání instance konfiguračního bloku pro MQTT připojení a přípravu kanálu na základě
\ic{NODE_ID}, resp. \ic{nodeId}.

Získaná instance bloku pro MQTT je zodpovědná za správu připojení -- není tedy nutné brát v potaz možné odpojení a
nutnost znovupřipojení -- tato instance bude následně použita k registraci odběrů a publikování zpráv do jednotlivých
MQTT kanálů protokolu.
Implementace tohoto bloku bohužel nemá zveřejněnou dokumentaci, veškeré poznatky o jeho funkci jsou tedy brány přímo
z jeho zdrojového kódu\footnote{\url{https://github
.com/node-red/node-red/blob/master/packages/node_modules/\%40node-red/nodes/core/io/10-mqtt.js}}.

% @formatter:off
\begin{code}[
    language=Javascript,
    label=code:fis-node-constructor,
    caption={Část konstruktoru třídy \ic{FisNode} obluhující připojení MQTT brokeru a přípravu kanálů pro komunikaci.}
]
class FisNode {
    constructor(config) {
        RED.nodes.createNode(this, config);
        this.broker = RED.nodes.getNode(config.broker);
        this.broker.connect(); // manually connect to MQTT broker
        // prepare topics for further usage
        this._publish_topic = ['fis', 'to', config.nodeId].join('/');
        this._subscribe_topic = ['fis', 'from', config.nodeId].join('/');
    }
    (*\ldots*)
    RED.nodes.registerType("fis-node", FisNode);
}
\end{code}
% @formatter:on

Druhá povinná část, šablona konfiguračního formuláře do editoru, je úzce spjata s registrací bloku -- jednotlivé
parametry musí zde odpovídat formulářovým prvkům, resp. jejím názvům, viz ukázka~\ref{code:fis-node-template}.
Dle konvence editoru Node-RED musí odpovídat jednotlivé atributy HTML elementů, aby byl editor schopen učinit
formulář interaktivním a zároveň i srozumitelným pro jádro editoru a získávání hodnot z něj.
V ukázce lze také vidět speciální způsob definice šablony -- šablona je obalena do elementu \ic{<script>} se
specifickou hodnotou v atributu \ic{type="text/x-red"}\footnote{Standard HTML při tomto chování označuje
element \ic{<script>} jako \uv{data block} a prohlížeče jsou povinny obsah elementů s tímto atributem dále
neinterpretovat.}.

% @formatter:off
\begin{code}[
    language=HTML,
    label=code:fis-node-template,
    caption={Ukázka z implementace druhé povinné části deklarace bloku -- šablona formuláře v jazyce HTML obsahuje
jednotlivé vstupní pro pole pro korespondující parametry definovené v registraci bloku do editoru
v ukázce~\ref{code:fis-node-constructor}.
Atribut \ic{id="node-config-input-broker"} (a odpovídající) jsou důležité vzhledem k chování editoru, nutná je shoda
s názvem parametru při registraci bloku -- stejně jako správné spárování šablony pomocí atributu
\ic{data-template-name="fis-node"}.},
]
<script type="text/x-red" data-template-name="fis-node">
    <div class="form-row" id="node-config-row-broker">
        <label for="node-config-input-broker">MQTT</label>
        <input type="text" id="node-config-input-broker">
    </div>
    (*\ldots*)
    <div class="form-row">
        <label for="node-config-input-nodeId">Node ID</label>
        <input type="text" id="node-config-input-nodeId">
    </div>
</script>
\end{code}
% @formatter:on

Vzhledem k aplikacím (a tedy již konkrétním blokům) poskytuje základní konfigurační blok několik metod pro práci s
MQTT komunikací -- z těch důležitějších se jedná o \ic{appPublish} a \ic{appSubscribe}.
Prvně jmenovaná metoda, zobrazená v ukázce~\ref{code:fis-node-app-publish}, slouží k publikování zprávy pro
konkrétní aplikaci (vlastnost požadovaná v kapitole~\ref{sec:pozadavky-na-protokol}) -- tedy do konkrétního kanálu.
Ten je v tomto místě složen z jednoznačné identifikace aplikace \ic{appId} (z návrhu odpovídá symbolu \ic{APP_ID}) a
volitelně podkanálu -- použitá privátní metoda \ic{_publish} poté ke kanálu před odesláním zprávy připojí identifikaci
uzlu a dojde tak k zacílení na konkrétní aplikaci na konkrétním uzlu.

% @formatter:off
\begin{code}[
    language=Javascript,
    label=code:fis-node-app-publish,
    caption={Detail z implementace třídy \ic{FisNode} -- metoda \ic{appPublish} poskytuje možnost konkrétnímu bloku
odeslání zprávy do odpovídající aplikace na uzlu.}
]
appPublish(appId, payload, subtopic = null) {
    return this._publish(
        // subtopic is optional, filter() to avoid double slash in channel
        ['app', appId, subtopic].filter(_ => _).join('/'),
        {
            payload,
            qos: payload.qos,
            retain: payload.retain,
        }
    );
};
\end{code}
% @formatter:on

Posledním zmínění-hodným detailem implementace třídy \ic{FisNode} je metoda pro odběr zpráv z aplikací na uzlech
\ic{appSubscribe}.
Tu mohou implementace jednotlivých bloků využít k zaregistrování odběru konkrétního kanálu, resp. konkrétní aplikace
-- výsledný kanál je složen ze základního kanálu (pro směr do nástroje Node-RED sestaveného již v konstruktoru),
identifikace aplikace \ic{appId} a volitelně podkanálu, opět dle navržené struktury protokolu z kapitoly
\ref{sec:mqtt-kanaly}.
Uložená instance připojení k MQTT brokeru poté slouží k zaregistrování samotného odběru -- ten je realizován pomocí
funkce, kterou pro každou příchozí zprávu MQTT klient invokuje.
S příchozími parametry je následně (po konverzi do formátu JSON) invokována funkce předaná při registraci konkrétního
odběru.

% @formatter:off
\begin{code}[
    language=Javascript,
    label=code:fis-node-app-subscribe,
    caption={Detail z implementace třídy \ic{FisNode} -- metoda \ic{appSubscribe} je určená k zaregistrování odběru
kanálu odpovídajícího konkrétní aplikaci na konkrétním uzlu.}
]
appSubscribe(appId, callback, subtopic = null, qos = 1, ref = 0) {
    const topic = [
        // subtopic is optional, filter() to avoid double slash in channel
        this._subscribe_topic, 'app', appId, subtopic
    ].filter(_ => _).join('/');
    return this.broker.subscribe(
        topic,
        qos,
        (topic, payload) => {
            callback(topic, JSON.parse(payload));
        },
        ref,
    );
};
\end{code}
% @formatter:on

\section{Aplikace a jejich správa na uzlu}
\todo{Vysledek}
\blind{3}

\section{Práce s~MQTT}
\todo{Jak poslouchat, jak publish atd.}
\blind{2}

\section{Detail implementace uzlu}
\todo{Jak se to da dal rozsirovat? a proc zrovna takhle?}
\blind{3}
