\chapter{Rozšíření pro Node-RED}
\label{ch:rozsireni}

Na základně poznatků z~kapitoly~\ref{sec:node-red-rozsireni} a navrženého komunikačního protokolu zde navrhnu a
zrealizuji\todo{weird} vlastní rozšíření nástroje Node-RED -- předmětem budou detaily ze základní infrastruktury bloků a
následně několik konkrétních
aplikací nasaditelných přímo v~reálném provozu.
% \marginpar{\small Poznámky na okraji}

\section{Základní konfigurační blok}\label{sec:zakladni-konfiguracni-blok}
Základním stavebním prvkem pro další bloky je \ic{fis-node} (reprezentovaný třídou \ic{FisNode}) -- využije se tak
konfiguračních bloků, díky kterým nebude nutné přihlašovací údaje k brokeru MQTT nebo identifikacím jednotlivých
uzlů zadávat více než jednou.

Jak uvádí dokumentace k nástroji Node-RED~\cite{NodeRedDocs}, nedělitelnou součástí každého bloku v Node-RED je jeho
konfigurační formulář pro editor.
Ten se skládá ze dvou povinných částí a jedné nepovinné -- z povinných částí jedná se o programovou definici pro editor
sítě v jazyce Javascript a o samotnou definici konfiguračního formuláře popsanou v jazyce HTML.
Nepovinnou částí je poté uživatelská dokumentace dostupná přímo z editoru.
Zkracenou definici bloku pro editor lze vidět v ukázce~\ref{code:fis-node-editor}, která kromě samotné registrace
obsahuje dva důležité aspekty.

Prvním je registrace do kategorie \ic{\'config\'} což značí registraci konfigurační bloku pro reprezentaci jednoho IoT
uzlu (uzel tedy nemá grafickou reprezentaci v síti).
Druhým aspektem je výčet parametrů pro formulář, které následně bude možné použít pro samotné chování uzlu v síti.
První parametr typu \ic{\'mqtt-broker\'} je určen pro konkrétní broker MQTT, pomocí kterého bude blok s uzlem spojen
(jedná se opět o konfigurační blok poskytnutý přímo nástrojem Node-RED).

Druhým parametrem je jednoznačná identifikace bloku, jejíž formát je omezen regulárním výrazem -- tento
parametr vychází z navrženého protokolu popsaného v kapitole~\ref{sec:mqtt-kanaly}\todo{čárka před "a"?} a to
konkrétně ze zástupného symbolu \ic{NODE\_ID}, který od sebe odlišuje jednotlivé uzly\footnote{Konkrétní tvar
identifikátoru (a jeho regulárního výrazu) není pro nástroj Node-RED podstatný, k jeho upřesnění dojde až rámci
firmwaru uzlů, popsaného v kapitole~\ref{ch:firmware}.}.
Parametr \ic{nodeId} tedy následně bude použit pro sestavení kanálů MQTT pro odběr a publikaci zpráv.

% @formatter:off
\begin{code}[
    language=Javascript,
    label=code:fis-node-editor,
    caption={Registrace vlastního bloku do editoru sítě v nástroji Node-RED.}
]
RED.nodes.registerType('fis-node', {
    category: 'config',
    defaults: {
        broker: {type: "mqtt-broker", required: true, value: ""},
        nodeId: {
            value: "", required: true,
            validate: RED.validators.regex(/^[a-z0-9]{12}(*\textdollar*)/i)
        },
    },
    (*\ldots*)
});
\end{code}

V ukázce~\ref{code:fis-node-constructor} je zobrazena část třídy \ic{FisNode}, reprezentující blok z hlediska sítě. V
konstruktoru této třídy dochází k získání konfiguračních
parametrů z formuláře, získání instance konfiguračního bloku pro připojení MQTT a přípravu kanálu na základě
\ic{NODE_ID}, resp. \ic{nodeId}.

Získaná instance bloku pro MQTT je zodpovědná za správu připojení -- není tedy nutné brát v potaz možné odpojení a
nutnost znovupřipojení -- tato instance bude následně použita k registraci odběrů a publikování zpráv do jednotlivých
MQTT kanálů protokolu.
Implementace tohoto bloku bohužel nemá zveřejněnou dokumentaci, veškeré poznatky o jeho funkci jsou tedy brány přímo
z jeho zdrojového kódu\footnote{\url{https://github
.com/node-red/node-red/blob/master/packages/node_modules/\%40node-red/nodes/core/io/10-mqtt.js}}.

% @formatter:off
\begin{code}[
    language=Javascript,
    numbers=left,
    label=code:fis-node-constructor,
    caption={Část konstruktoru třídy \ic{FisNode} obluhující připojení brokeru MQTT a přípravu kanálů pro komunikaci.}
]
class FisNode {
    constructor(config) {
        RED.nodes.createNode(this, config);
        this.broker = RED.nodes.getNode(config.broker);
        this.broker.connect(); // manually connect to MQTT broker
        // prepare topics for further usage
        this._publish_topic = ['fis', 'to', config.nodeId].join('/');
        this._subscribe_topic = ['fis', 'from', config.nodeId].join('/');
    }
    (*\ldots*)
    RED.nodes.registerType("fis-node", FisNode);
}
\end{code}
% @formatter:on

\todo{Ještě trochu o FisNode a jeho kontrole statusu}

Druhá povinná část, šablona konfiguračního formuláře do editoru, je úzce spjata s registrací bloku -- jednotlivé
parametry musí zde odpovídat formulářovým prvkům, resp. jejím názvům, viz ukázka~\ref{code:fis-node-template}.
Dle konvence editoru Node-RED musí odpovídat jednotlivé atributy HTML elementů, aby byl editor schopen učinit
formulář interaktivním a zároveň i srozumitelným pro jádro editoru a získávání hodnot z něj.
V ukázce lze také vidět speciální způsob definice šablony -- šablona je obalena do elementu \ic{<script>} se
specifickou hodnotou v atributu \ic{type="text/x-red"}\footnote{Standard HTML při tomto chování označuje
element \ic{<script>} jako \uv{data block} a prohlížeče jsou povinny obsah elementů s tímto atributem dále
neinterpretovat.}.

% @formatter:off
\begin{code}[
    language=HTML,
    label=code:fis-node-template,
    caption={Ukázka z implementace druhé povinné části deklarace bloku -- šablona formuláře v jazyce HTML obsahuje
    jednotlivé vstupní pro pole pro korespondující parametry definovené v registraci bloku do editoru
    v ukázce~\ref{code:fis-node-constructor}.
    Atribut \ic{id="node-config-input-broker"} (a odpovídající) jsou důležité vzhledem k chování editoru, nutná je shoda
    s názvem parametru při registraci bloku -- stejně jako správné spárování šablony pomocí atributu
    \ic{data-template-name="fis-node"}.},
]
<script type="text/x-red" data-template-name="fis-node">
    <div class="form-row" id="node-config-row-broker">
        <label for="node-config-input-broker">MQTT</label>
        <input type="text" id="node-config-input-broker">
    </div>
    (*\ldots*)
    <div class="form-row">
        <label for="node-config-input-nodeId">Node ID</label>
        <input type="text" id="node-config-input-nodeId">
    </div>
</script>
\end{code}
% @formatter:on

Vzhledem k aplikacím (a tedy již konkrétním blokům) poskytuje základní konfigurační blok několik metod pro práci s
MQTT komunikací -- z těch důležitějších se jedná o \ic{appPublish} a \ic{appSubscribe}.
Prvně jmenovaná metoda, zobrazená v ukázce~\ref{code:fis-node-app-publish}, slouží k publikování zprávy pro
konkrétní aplikaci (vlastnost požadovaná v kapitole~\ref{sec:pozadavky-na-protokol}) -- tedy do konkrétního kanálu.
Ten je v tomto místě složen z jednoznačné identifikace aplikace \ic{appId} (z návrhu odpovídá symbolu \ic{APP_ID}) a
volitelně podkanálu -- použitá privátní metoda \ic{_publish} poté ke kanálu před odesláním zprávy připojí identifikaci
uzlu a dojde tak k zacílení na konkrétní aplikaci na konkrétním uzlu.

% @formatter:off
\begin{code}[
    language=Javascript,
    label=code:fis-node-app-publish,
    caption={Detail z implementace třídy \ic{FisNode} -- metoda \ic{appPublish} poskytuje možnost konkrétnímu bloku
    odeslání zprávy do odpovídající aplikace na uzlu.}
]
appPublish(appId, payload, subtopic = null) {
    return this._publish(
        // subtopic is optional, filter() to avoid double slash in channel
        ['app', appId, subtopic].filter(_ => _).join('/'),
        {
            payload,
            qos: payload.qos,
            retain: payload.retain,
        }
    );
};
\end{code}
% @formatter:on

Posledním zmínění-hodným detailem implementace třídy \ic{FisNode} je metoda pro odběr zpráv z aplikací na uzlech
\ic{appSubscribe}.
Tu mohou implementace jednotlivých bloků využít k zaregistrování odběru konkrétního kanálu, resp. konkrétní aplikace
-- výsledný kanál je složen ze základního kanálu (pro směr do nástroje Node-RED sestaveného již v konstruktoru),
identifikace aplikace \ic{appId} a volitelně podkanálu, opět dle navržené struktury protokolu z kapitoly~\ref{sec:mqtt-kanaly}.
Uložená instance připojení k brokeru MQTT poté slouží k zaregistrování samotného odběru -- ten je realizován pomocí
funkce, kterou pro každou příchozí zprávu MQTT klient invokuje.
S příchozími parametry je následně (po konverzi do formátu JSON) invokována funkce předaná při registraci konkrétního
odběru.

% @formatter:off
\begin{code}[
    language=Javascript,
    label=code:fis-node-app-subscribe,
    caption={Detail z implementace třídy \ic{FisNode} -- metoda \ic{appSubscribe} je určená k zaregistrování odběru
    kanálu odpovídajícího konkrétní aplikaci na konkrétním uzlu.
    Parametr \ic{qos} slouží k nastavení konkrétní QoS pro tento odběr, \ic{ref} je poté volitelná identifikace odběru,
    lze pomocí ní poté mazat konkrétní odběry.}
]
appSubscribe(appId, callback, subtopic = null, qos = 1, ref = 0) {
    const topic = [
        // subtopic is optional, filter() to avoid double slash in channel
        this._subscribe_topic, 'app', appId, subtopic
    ].filter(_ => _).join('/');
    return this.broker.subscribe(
        topic,
        qos,
        (topic, payload) => {
            callback(topic, JSON.parse(payload));
        },
        ref,
    );
};
\end{code}
% @formatter:on

\section{Implementace bloku pro vstupní aplikaci}\label{sec:implementace-bloku-pro-vstupni-aplikaci}
Blok pro vstupní aplikaci je z pohledu nástroje Node-RED blokem, který do sítě produkuje zprávy (přijaté z třetí
strany).
Implementace bloku pro konkrétní aplikaci vychází ze základního konfiguračního bloku -- s využitím pomocné funkce
nástroje Node-RED \ic{RED.nodes.getNode} lze získat instanci bloku, která reprezentuje IoT uzel připojený přes MQTT
kanály.
V ukázce~\ref{code:fis-dht-app} lze ve spodní části těla konstruktoru vidět dvě volání -- nad konfiguračním blokem se
prvně volá metoda \ic{config}.
Tato metoda zajišťuje napojení na servisní aplikaci určenou pro správu uzlu jako takového a dalších aplikací --
detaily její funkce budou popsány v kapitole~\ref{ch:firmware}.
Volání konkrétně odešle do uzlu uživatelské nastavení společně s identifikátorem aplikace \ic{\'dht-sensor\'} -- zahrnut
je typ senzoru a pin, na kterém je na uzlu připojen.
Pro konkrétní spárování dotčeného bloku s aplikací na uzlu je odeslán i atribut bloku \ic{this.id}, který slouží v
rámci nástroje Node-RED k unikátní identifikaci uzlu v celé síti\footnote{Atribut \ic{id} instancí bloků je
odsledován z interní implementace běhového prostředí Node-RED -- \url{https://github
.com/node-red/node-red/blob/ed2a45e97551d9e43f079d69be8a490574e98559/packages/node_modules/\%40node-red/runtime/lib
/nodes/flows/Subflow.js}}, generován je náhodně.

Druhé volání je použití již popsané metody \ic{appSubscribe}, blok si zde registruje odběr na data přicházející z
uzlu.
Používá k tomu vlastní atribut \ic{id}, kterým cílí na spárovanou aplikaci na uzlu, podkanál \ic{\'data\'}, který
následně použije i aplikace na uzlu, a funkci, pomocí které bude předávat příchozí zprávu z uzlu, resp. brokeru MQTT,
do sítě nástroje Node-RED (skrz volání metody \ic{this.send}).

% @formatter:off
\begin{code}[
    language=Javascript,
    label=code:fis-dht-app,
    caption={Detail implementace vstupní aplikace (z hlediska centrálního uzlu) -- konkrétně se jedná o aplikaci pro
    senzory typu DHT měřící teplotu a vlhkost okolí. }
]
class DhtSensor {
    constructor(config) {
        RED.nodes.createNode(this, config);
        this.fisNode = RED.nodes.getNode(config.node);

        // send app configuration to node
        this.fisNode.config('dht-sensor', this.id, {
            port: config.sensorPort,
            type: config.sensorType,
        });
        // subscribe data from node
        this.fisNode.appSubscribe(this.id, (topic, payload) => {
            this.send({
                temperature: payload.temperature,
                humidity: payload.humidity
            });
        }, 'data');
    }
}
\end{code}
% @formatter:on

\section{Implementace bloku pro výstupní aplikaci}\label{sec:implementace-bloku-pro-vystupni-aplikace}
Blok pro výstupní aplikaci je z pohledu nástroje Node-RED blokem, který ze sítě konzumuje zprávy (a distribuuje je
dále třetím stranám).
V ukázce~\ref{code:fis-node-neopixel-display} lze vidět implementaci bloku pro aplikaci ovládající bodový displej,
jejíž detaily budou popsány v kapitole~\ref{ch:firmware}.
Po standardním zavedením cílové aplikace na uzel si blok registruje odběr pomocí vlastní metody \ic{on} -- ta slouží
k odběru příchozí zpráv do bloku, v tomto případě převezme příchozí zprávu a zformátuje zprávu pro aplikaci
(kromě samotného textu k zobrazení podporuje volitelně i požadovanou barvu).
Sestavená zpráva je odeslána pomocí \ic{appPublish} odpovídající aplikaci do subkanálu \ic{\'text\'}.

% @formatter:off
\begin{code}[
    language=Javascript,
    label=code:fis-node-neopixel-display,
    caption={Implementace bloku pro aplikaci ovládající bodový displej -- kromě samotné konfigurace na cílovém uzlu
    si uzel zaregistruje funkci pro odběr události typu \ic{\'input\'}.
    Událost tohoto typu notifikuje blok o příchozí zprávě, které je v tomto případě odeslána do aplikace k zobrazení
    na displeji.}
]
class NeoPixelDisplay {
    constructor(config) {
        RED.nodes.createNode(this, config);
        this.fisNode = RED.nodes.getNode(config.node);

        this.fisNode.config('neopixel-display', this.id, {
            port: config.displayPort,
            width: config.width,
            height: config.height,
        });

        this.on('input', msg => {
            let payload = {text: msg.payload};

            if (msg.color)
                payload.color = msg.color;
            payload.retain = true;

            this.fisNode.appPublish(this.id, payload, 'text');
        });
    }
}
\end{code}
% @formatter:on

\section{Detail implementace uzlu}
\todo{Jak se to da dal rozsirovat? a proc zrovna takhle?}
\blind{3}
