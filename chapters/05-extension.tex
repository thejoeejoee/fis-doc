\chapter{Rozšíření pro Node-RED}
\label{ch:rozsireni}

Na základně poznatků z~kapitoly~\ref{sec:node-red-rozsireni} a komunikačního protokolu zde navrhnu a zrealizuji vlastní
rozšíření nástroje Node-RED -- předmětem bude základní infrastruktura bloků a poté několik konkrétních aplikací
nasaditelných přímo v~reálném provozu.
% \marginpar{\small Poznámky na okraji}

\section{Základní konfigurační blok}\label{sec:zakladni-konfiguracni-blok}
Základním stavebním prvkem pro další bloky je \ic{fis-node} (reprezentovaný třídou \ic{FisNode}) -- využije se tak
konfiguračních bloků, díky kterým
nebude nutné přihlašovací údaje k MQTT brokeru nebo identifikacím jednotlivých uzlů zadávat více než jednou.
V konstruktoru této třídy, zobrazeném v ukázce~\ref{code:fis-node-constructor}, dochází k získání konfiguračních
parametrů z formuláře (bude popsán níže), získání instance konfiguračního bloku pro MQTT připojení a přípravu kanálu
na základě \ic{NODE_ID}.

Získaná instance bloku pro MQTT je zodpovědná za správu připojení -- není tedy nutné brát v potaz možné odpojení a
nutnost znovupřipojení -- tato instance bude následně použita k registraci odběrů a publikování zpráv do jednotlivých
MQTT kanálů protokolu.
Implementace tohoto bloku bohužel nemá zveřejněnou dokumentaci, veškeré poznatky o jeho funkci jsou tedy brány přímo
z jeho zdrojového kódu\footnote{\url{https://github
.com/node-red/node-red/blob/master/packages/node_modules/\%40node-red/nodes/core/io/10-mqtt.js}}.

% @formatter:off
\begin{code}[
    numbers=left,
    language=Javascript,
    label=code:fis-node-constructor,
    caption={Část konstruktoru třídy \ic{FisNode} obluhující připojení MQTT brokeru a přípravu kanálů pro komunikaci.}
]
class FisNode {
    constructor(config) {
        RED.nodes.createNode(this, config);
        this.broker = RED.nodes.getNode(config.broker);
        this.broker.connect(); // manually connect to MQTT broker
        // prepare topics for further usage
        this._publish_topic = ['fis', 'to', config.nodeId].join('/');
        this._subscribe_topic = ['fis', 'from', config.nodeId].join('/');
    }
    #!\ldots
    RED.nodes.registerType("fis-node", FisNode);
}
\end{code}

Nedělitelnou součástí každébo bloku v Node-RED je jeho konfigurační formulář.

\begin{code}[
    numbers=left,
    language=HTML,
]
<script type="text/javascript">
    RED.nodes.registerType('fis-node', {
        category: 'config',
        defaults: {
            broker: {type: "mqtt-broker", required: true, value: ""},
            nodeId: {value: "", required: true, validate: RED.validators.regex(/^[a-z0-9]{12}$/i)},
            name: {},
        },
        label: function () {
            return this.name || this.broker;
        }
    });
</script>

<script type="text/x-red" data-template-name="fis-node">
    <div class="form-row" id="node-config-row-broker">
        <label for="node-config-input-broker">MQTT</label>
        <input type="text" id="node-config-input-broker">
    </div>
    <div class="form-row">
        <label for="node-config-input-nodeId">Node ID</label>
        <input type="text" id="node-config-input-nodeId">
    </div>
    <div class="form-row">
        <label for="node-config-input-name">Name</label>
        <input type="text" id="node-config-input-name">
    </div>
</script>
\end{code}
% @formatter:on

\section{Aplikace a jejich správa na uzlu}
\todo{Vysledek}
\blind{3}

\section{Práce s~MQTT}
\todo{Jak poslouchat, jak publish atd.}
\blind{2}

\section{Detail implementace uzlu}
\todo{Jak se to da dal rozsirovat? a proc zrovna takhle?}
\blind{3}
