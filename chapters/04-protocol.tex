\chapter{Návrh komunikačního protokolu}
\label{ch:protokol}

Základním spojovacím prvkem pro rozšíření nástroje Node-RED a vlastních uzlů je rozhraní, pomocí kterého budou
komunikovat.
S využitím protokolu MQTT a jeho vlastností popíšu v této kapitole základní požadavky na komunikační protokol a
následně jej navrhnu -- ze základních požadavků poté budou plynout i požadavky na obě strany komunikace (uzly i
Node-RED).

\section{Požadavky na protokol}\label{sec:pozadavky-na-protokol}
Na základě povahy obou komunikujících stran jsem navrhnul tyto základní požadavky:
\begin{enumerate}
    \item \textbf{Pro formát zpráv je použit JSON} \\
    Důvodem je jeho jednoduchost, kvalitní podpora a textová podstata (tedy i snažší ladění).
    Ze strany Node-RED je tento formát implicitní -- jedná se o \uv{Javascript Object Notation} -- serializace a
    deserializace probíhají zcela přirozeně.
    Na straně uzlů pak poskytuje MicroPython kompletní podporu pro tento formát -- deserializace probíhá do
    vestavěných typů.

    \item \textbf{Samostatné kanály pro směry \uv{do uzlu} a \uv{z uzlu}} \\
    Pro snížení datového toku a cílení zpráv pro konkrétní uzly je nutné oddělit komunikační kanály pro každý
    samostatný uzel -- rozšíření na straně Node-RED zajistí směřování do konkrétních kanálů dle konfigurace a uzly
    naopak odběr kanálů pouze příslušících danému uzlu.

    \item \textbf{Samostatné kanály pro jednotlivé aplikace na uzlu} \todo{popsat aplikace na uzlech někdy dřív} \\
    Jednotlivé aplikace běžící na uzlech je nutné v protokolu od sebe oddělovat -- tzn. kromě rozlišení na úrovni
    všech uzlů musí dojít k rozlišení běžících aplikací na úrovni jednoho uzlu.
\end{enumerate}


\section{MQTT kanály}
\todo{Využití MQTT kanálů \& retain}
\blind{1}


\section{Navržený protokol}
\todo{Vysledek}
\blind{3}

\missingfigure{Schema komunikace}