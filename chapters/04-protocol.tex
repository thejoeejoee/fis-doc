\chapter{Návrh komunikačního protokolu}
\label{ch:protokol}

Základním spojovacím prvkem pro rozšíření nástroje Node-RED a vlastních uzlů je rozhraní, pomocí kterého budou
komunikovat.
S využitím protokolu MQTT a jeho vlastností popíšu v této kapitole základní požadavky na komunikační protokol a
následně jej navrhnu -- ze základních požadavků poté budou plynout i požadavky na obě strany komunikace (uzly i
Node-RED -- centrální uzel).
\todo{Vzhledem k možnostem jsem se rozhodl koncové uzly používat víceúčelově -- tedy aplikace atd.}

\section{Požadavky na protokol}\label{sec:pozadavky-na-protokol}
Na základě povahy a potřeb obou komunikujících stran jsem navrhnul tyto základní požadavky:
\begin{enumerate}
    \item \textbf{Pro zprávy je použit formát JSON} \\
    Důvodem je jeho jednoduchost, kvalitní podpora a textová podstata (tedy i snažší ladění).
    Ze strany Node-RED je tento formát implicitní -- jedná se o \uv{Javascript Object Notation} -- serializace a
    deserializace probíhají zcela přirozeně.
    Na straně uzlů pak poskytuje MicroPython kompletní podporu pro tento formát -- deserializace probíhá do
    vestavěných typů.

    \item \textbf{Samostatné kanály pro směry \uv{do uzlu} a \uv{z uzlu}} \\
    Pro snížení datového toku a cílení zpráv pro konkrétní uzly je nutné oddělit komunikační kanály pro každý
    samostatný uzel -- rozšíření na straně Node-RED zajistí směřování do konkrétních kanálů dle konfigurace a uzly
    naopak odběr kanálů pouze příslušících danému uzlu.

    \item \textbf{Samostatné kanály pro jednotlivé aplikace na uzlu} \todo{popsat aplikace na uzlech někdy dřív} \\
    Jednotlivé aplikace běžící na uzlech je nutné v protokolu od sebe oddělovat -- tzn. kromě rozlišení na úrovni
    všech uzlů musí dojít k rozlišení běžících aplikací na úrovni jednoho uzlu.

    \item \textbf{Kanály pro status a konfiguraci uzlu} \\
    Pro režijní komunikaci s uzlem je nutný samostatný konfigurační kanál, pomocí kterého bude uzel příjmat příkazy k
    zavedení aplikace, její překonfigurování či ukončení -- dále pomocí něj mohou být přenášeny informace o vytížení
    uzlu, chybách aplikací či nastavení připojení k Internetu.
    Druhým nutným kanálem je zpětný kanál statusu, kterým bude uzel oznamovat do sítě svůj stav (připojen či
    nepřipojen), případně další informace (využití úložiště, čekající data).
\end{enumerate}

\section{MQTT kanály}\label{sec:mqtt-kanaly}
V popisu kanálů budou použity následující symboly:
\begin{itemize}
    \item \ic{NODE_ID} je jednoznačná identifikace uzlu v síti \todo{referenci na mac adresu esp}
    \item \ic{APP_ID} je jednoznačná identifikace instance aplikace v rámci uzlu
    \item \ic{\#} je zástupný znak popsaný v kapitole~\ref{sec:message-queuing-telemetry-transport}
\end{itemize}

Na základě požadavků stanovených v kapitole~\ref{sec:pozadavky-na-protokol} jsem navrhnul použití následujících kanálů
MQTT (funkce budou popsány z pohledu centrálního uzlu):

\begin{itemize}
    \item \ic{fis/to/NODE_ID/app/APP_ID} \\
    Kanál je určen pro zasílání zpráv do konkrétní aplikace na uzlu -- a navíc,
    všechny jeho podkanály (\ic{fis/to/NODE_ID/app/APP_ID/\#}) také míří do dané aplikace.

    \item \ic{fis/from/NODE_ID/app/APP_ID} \\
    Tento kanál je určen pro odběr zpráv z konkrétní aplikace na uzlu --
    všechny jeho podkanály (\ic{fis/from/NODE_ID/app/APP_ID/\#}) také míří z dané aplikace.

    \item \ic{fis/from/NODE_ID/status} \\
    Slouží pro signalizaci stavu uzlu pro Node-RED -- s pomocí parametru zpráv \uv{retain} (popsaného v
    kapitole~\ref{subsec:priznak-zpravy-retain}), zde dochází k zachování posledního známého stavu uzlu.
\end{itemize}



\todo{zapnout JSON pro IC}
\begin{table}
    \centering

    \caption{Příklady využití navrhnutého prokolu pro komunikaci -- řádek vždy představuje jednu konkrétní zprávu v
    protokolu MQTT}
    \begin{tabularx}{\textwidth}{s|l|s}
        \textbf{kanál} & \textbf{zpráva} & \textbf{význam} \\
        \hline
        \ic{fis/from/25a8ced/status} & \ic{\{"online": true\}} & uzel \uv{25a8ced} oznamuje změnu svého stavu -- své
        připojení \\

        \ic{fis/from/25a8ced/app/74cae6f} & \ic{\{"value": 0.3745\}} & aplikace publikuje obecná data bez rozlišení
        subkanálu\\

        \ic{fis/from/25a8ced/app/74cae6f/temperature} & \ic{\{"value": 18.9\}} & aplikace publikuje data do svého
        subkanálu \uv{temperature} \\

        \ic{fis/to/25a8ced/app/config} & \ic{\{"action": "reload"\}} & centrální uzel zasílá zprávu pro aplikaci
        \uv{config} \\

        \ic{fis/to/25a8ced/app/3b1cef/bottom} & \ic{\{"color": "orange"\}} & centrální uzel zasílá aplikaci
        \uv{3b1cef} zprávu do jejího subkanálu \uv{bottom} \\


    \end{tabularx}
    \label{table:mqtt-subscribes}
\end{table}


\todo{Využití MQTT kanálů \& retain}
\blind{1}


\section{Navržený protokol}
\todo{Vysledek}
\blind{3}

\missingfigure{Schema komunikace}