\chapter{Závěr}
\label{ch:zaver}

Cílem této práce bylo za pomocí nástroje Node-RED navrhnout a realizovat prostředky pro koordinaci uzlů Internetu věcí
založených na čipech ESP32 s programovým vybavením interpretu MicroPython.
Na straně nástroje pro koordinaci Internetu věcí je možné nainstalovat rozšíření zajištující reprezentaci fyzických
uzlů v tomto nástroji
-- síť je poté schopna skrz toto rozšíření komunikovat s dynamicky nakonfigurovanými aplikacemi na uzlech.
To je zajištěno vlastním protokolem nad kanály brokeru MQTT, ke kterému jsou nástroj Node-RED a fyzické uzly připojeny.
Pro uzly na bázi čipu ESP32 je v jazyce MicroPython zrealizován firmware, jenž obsahuje asynchronní smyčku
obsluhující komunikaci s brokerem a jednotlivé nakonfigurované aplikace.
Důležitou vlastností takto zrealizovaného systému je jeho rozšířitelnost -- pro další aplikaci stačí pouze implementace
ovladače některé z periférií čipu ESP32 a protistrana ve formě bloku do editoru nástroje Node-RED.

Pro demonstraci funkce jsou naimplementovány vzorové aplikace různých druhů, které jsou poté použity pro ověření
funkce v rámci celé sítě -- to je zajištěno pomocí testovacích scénářů popsaných v závěrečné kapitole.