\chapter{Micropython firmware pro ESP32}
\label{ch:firmware}

V této kapitole bude popsán návrh a implementace řídícího softwaru, tzv. firmwaru pro čipy ESP32 -- od základního
jádra zodpovědného za zavedení aplikací či připojení k brokeru MQTT až k aplikacím a jejich asynchronnímu principu.

\section{Základní požadavky pro jádro}\label{sec:základní-požadavky-pro-jádro}
Na základě navrženého protokolu a vlastního úsudku jsem navrhl následující požadavky na implementaci jádra firmwaru:

\begin{itemize}
    \item Jádro je koncový bod pro komunikační kanál s Node-RED\\
    \item
\end{itemize}



\subsection{Využití režimu nízkého odběru}
\todo{Uspavani jadra, odber}
\blind{2}

\section{Aplikace}
\todo{Princip aplikaci, nodered-node === aplikace}
\blind{2}

\subsection{Možnosti aplikace}
\todo{Rozbor moznosti aplikace}
\blind{2}

\subsection{Detail implementace aplikace}
\todo{Jedna nazorna aplikace}
\blind{3}

