\chapter{Úvod}
\label{ch:uvod}

Internet věcí je v~roce 2019 pokračujícím fenoménem nejen pro odborníky z~oblasti automatizace či vestavěných systémů.
Díky dostupným prostředkům ve formě mikrokontrolérů, \textit{System on Chip} řešeních či komunikačních modulů není
v~dnešní době problém realizovat monitorování nebo automatizaci domácnosti, kanceláře nebo podniku.

Experti odhadují, že do roku 2020 bude do sítě Internet připojeno až \textbf{31 miliard IoT}
zařízení~\cite{StatistaIoT, IoTTrends}, za dalších pět let poté víc jak dvojnásobný počet.
Je tedy na místě se ptát na otázky dostupnosti programového vybavení pro konkrétní IoT uzly, jejich koordinaci a sběr
a agregaci dat ze sítí.
Také je nutné brát v~potaz uživatelskou přívětivost těchto nástrojů, vzhledem ke
skutečnosti rozšíření internetu věcí mezi širokou odbornou, či alespoň poučenou laickou, veřejnost.

V~následujících kapitolách bude představen návrh a realizace komplexního rozšíření nástroje Node-RED použitého pro
koordinaci uzlů sítě IoT. Jako uzly poslouží zařízení založená na ESP32, pro které bude navržen a implementován
operační systém a protokol pro komunikaci.
Cílem práce je vytvoření \textbf{otevřené platformy} pro koordinaci IoT uzlů nad nástrojem Node-RED zahrnující
programové rozšíření tohoto nástroje a firmware pro uzly ve formě čipů ESP32 -- cílem práce naopak není
detailní rozbor hardwarového řešení pro uzly sítě Internetu věcí.

% https://www.itproportal.com/features/next-big-things-in-iot-predictions-for-2020/
% https://www.newgenapps.com/blog/iot-trends-how-internet-of-things-will-evolve-by-2020
% https://www.statista.com/statistics/471264/iot-number-of-connected-devices-worldwide/