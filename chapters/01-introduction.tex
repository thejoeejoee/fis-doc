\chapter{Úvod}
\label{ch:uvod}

Internet věcí je v roce 2019 pokračujícím fenoménem nejen pro odborníky z oblasti automatizace či vestavěných systémů. Díky dostupným prostředkům ve formě mikrokontrolérů, SoC řešení či komunikačních modulů není v dnešní době překážkou zrealizovat monitorování nebo automatizaci domácnosti, kanceláře nebo podniku.

Experti odhadují, že do roku 2020 bude do sítě Internet připojeno až \textbf{31 miliard IoT} zařízení, za dalších pět let poté víc jak dvojnásobný počet.
% https://www.itproportal.com/features/next-big-things-in-iot-predictions-for-2020/
% https://www.newgenapps.com/blog/iot-trends-how-internet-of-things-will-evolve-by-2020
% https://www.statista.com/statistics/471264/iot-number-of-connected-devices-worldwide/

Je tedy na místě se ptát na otázky dostupnosti programového vybavení pro konkrétní IoT uzly, jejich koordinaci a sběr a agregaci dat ze sítí. Také není možné nebrát v potaz uživatelskou přívětivost těchto nástrojů, vzhledem ke skutečnosti rozšíření internetu věcí mezi širokou odbornou, či alespoň laickou zainterestovanou, veřejnost.

\cite{ESP32Datasheet}
\section{\emph{Internet of Things}}
\blind{1}

\todo{Barman}