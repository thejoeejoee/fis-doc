\chapter{Principy dnešního internetu věcí}
\label{ch:principy-iot}

\blind{1}

\todo{Co je to IoT a proc jej potrebujeme?}
\blind{1}

\missingfigure{Schema centralizovaneho iot systemu}

\subsection{Bezpečnost systémů}
\todo{Duraz na bezpecnost a uzivatelskou privetivost}
\blind{2}

\todo{Rozdelit na SW a HW casti}
\blind{1}

\section{Existující řešení}
\subsection{Hardware}
\todo{Pro SW poresit Loxone/SmartHome}
\blind{1}

\subsection{Software}
\todo{Pro HW poresit komercni site/pyboardy a podobne}
\blind{1}

\section{ESP32}
\textit{ESP32} je řada nízkonákladových SoC\footnote{\textit{System on Chip} je integrovaný obvod obsahující veškeré periferie (digitální, analogové a často i rádiová rozhraní) zabudované přímo v čipu. Tento princip bývá použit e vestavěných systémech díky jeho nízdé spotřebě.} mikrokontrolérů představená čínskou firmou Espressif Systems. Jakožto nástupce řady \textit{ESP8266} nabízí na jediném čipu následující:

\begin{itemize}
    \item 32 bitový mikroprocesor Xtensa LX6, taktovaný na 160 či 240 MHz
    \item 520 KiB statické operační paměti
    \item Wi-Fi ve stadardu 802.11 b/g/n
    \item Bluetooth v4.2 včetně nízkoodběrového režimu BLE
    \item 12 bitový ADC až s 18 kanály
    \item dva 2 bitové DAC
    \item čtyři rozhraní SPI
    \item tři rozhraní UART
    \item deset GPIO pinů pro kapacitní použití
    \item rozhraní I\textsuperscript{2}C, I\textsuperscript{2}S, Hallovu sondu, sběrnici CAN, generátory PWM a další
\end{itemize}

\todo{ESP32}
\blind{3}
\missingfigure{ESP32}

\subsection{Měření neelektrických veličin}
\todo{Co vsechno se meri - kratky uvod do HW principu mereni?}
\blind{1}

\section{MicroPython}
\todo{uPython}
MicroPython je derivátem vysokoúrovňového skriptovacího programovacího jazyka Python určený pro běh na vestavěných systémech a dalších
aplikacích s nízkým výpočetním výkonem. Z hlediska vestavěné knihovny nabízí naprostou většinu základní knihovny z původní
distribuce a navíc knihovny zodpovědné za manipulaci s rozhraním mikropočítače, na kterém běží. Kromě oficiálního sestavení
pro pyboard\footnote{Oficiální (výukový) vývojový kit pro běh Micropythonu s cenou okolo \pounds30 z oficiálního e-shopu.}
nabízí komunita i sestavení pro ESP2866, ESP32, WiPy nebo Espruino Pico.

Sestavení pro kity od společnosti ESP jsou založena na vývojovém frameworku ESP-IDF.
Ten je zástupcem operačních systémů reálného času (RTOS), tedy systémů pro které je typické časové plánování jednotlivých úloh
a kritické jejich časově přesné spuštění -- to vše nzaložené na plánovači, který bývá napostradatelnou součástí. ESP-IDF
je konkrétně postaven nad FreeRTOS, který je de facto standardem a největším hráčem na poli RTOS s otevřeným zdrojovým kódem. \todo{citace z https://www.simform.com/iot-rtos-selection/}
Jedním z benefitů tohoto typu systému je možnost použití programového řízení komunikačních rozhraní
(jako je SPI, I\textsuperscript{2}C nebo USART) -- tato rozhraní lze poté naimplementovat, není nutný konkrétní hardwarový prvek.

I přes omezenou vestavěnou knihovnu a vysokoúrovňovost tohoto jazyka jej lze použít ve světě internetu věcí --
a to především díky jeho paměťové optimalizaci. Nespornou výhodou pro použití v internetu věcí je i dostupnost základních knihoven
pro komunikaci s okolním světem. MicroPython v základu nabízí knihovnu \ic|machine| s třídami \ic|Pin|, \ic|PWM| či \ic|ADC| --
každá z těchto tříd reprezentuje způsob, jak komunikovat s okolním světem pomocí vestavěných periférií. I vestavěná knihovna
je ovšem svým obsahem závislá na konkrétní distribuci pro konkrétní vývojový kit

\begin{code}
    import machine
    pin = machine.Pin(14, mode=machine.Pin.OUT)
    pin.value(1)
\end{code}


\section{Message Queuing Telemetry Transport}\label{sec:message-queuing-telemetry-transport}
Message Queuing Telemetry Transport (MQTT) je protokol definovaný ISO standardem určený pro kanálovou komunikaci zařízení
mezi sebou. Jedná se o typ komunikace klient-server nad protokolem TCP/IP, server je dle svých vlastností nazýván specificky
jako \uv{broker}. Klient má možnost si po připojení zaregistrovat odběry jednotlivých kanálů -- jejich identifikace má tvar
alfanumerických řetězců oddělených pomocí znaku \ic|/|. Benefitem přístupu s jasně definovaným oddělovačem je možnost použití zástupných znaků pro
zaregistrování odběru více kanálů zároveň. Této vlastnosti bude využito v pozdější části práce pro komunikaci s uzly. Znak \ic|+|
je použit pro zastoupení jedné úrovně kanálu, znak \ic|#| pro víceúrovňové zahrnutí -- demonstrace těchto principů je shrnuta v
tabulce~\ref{table:02:mqtt-subscribes}.

Kromě základního nabízí protokol MQTT ještě několik dalších možností:

\begin{itemize}
    \item \textbf{Příznak zprávy \uv{retain}} \\
        Pomocí příznaku \uv{retain} může klient pro zprávu v konkrétním kanálu nastavit její ponechání v kanálu pro budoucí klienty
        -- tedy, zaregistruje-li si klient kanál k odběru, broker klientovi automaticky po zaregistrování odešle zprávy,
        spadající tohoto kanálu (či kanálů v případě zástupných znaků), které mají nastavený tento příznak. Tohoto chování lze
        velmi dobře využít pro kanály obsahující stav kterékoliv části ze systému -- nově připojený klient pak ihned dostane zprávu
        o stavu a nedochází k časovému intervalu, kdy je sice klient zaregistrovaný, ale teprve čeká na novou aktualizaci stavu.

    \item
\end{itemize}

\begin{table}
    \centering
    \begin{tabularx}{\textwidth}{ssr}
        \textbf{kanál zaregistrovaný k odběru} & \textbf{kanál zprávy} & \textbf{bude zpráva zahrnuta?} \\
        \hline
            \emph{node/1}
            &
            \emph{node/1} \newline
            \emph{node} \newline
            \emph{node/1/status} \newline
            \emph{node/2}
            &
            \truemark \newline
            \falsemark \newline
            \falsemark \newline
            \falsemark
        \\

        \hline
            \emph{node/+/status}
            &
            \emph{node/1/status} \newline
            \emph{node/2/status} \newline
            \emph{node/foo/status} \newline
            \emph{node/1/data} \newline
            \emph{node} \newline
            \emph{node/2}
            &
            \truemark \newline
            \truemark \newline
            \truemark \newline
            \falsemark \newline
            \falsemark \newline
            \falsemark
        \\

        \hline
            \emph{node/\#}
            &
            \emph{node/1} \newline
            \emph{node/2/status} \newline
            \emph{node/2/status/data} \newline
            \emph{node/} \newline
            \emph{block/}
            &
            \truemark \newline
            \truemark \newline
            \truemark \newline
            \falsemark \newline
            \falsemark
        \\

        \hline
            \emph{node/+/data/\#}
            &
            \emph{node/1/data/value} \newline
            \emph{node/2/data/value/degrees} \newline
            \emph{node/1/data/value} \newline
            \emph{node/1/status/value} \newline
            \emph{node/} \newline
            \emph{block/}
            &
            \truemark \newline
            \truemark \newline
            \truemark \newline
            \falsemark \newline
            \falsemark \newline
            \falsemark
        \\

    \end{tabularx}
    \caption{Možnosti odběru zpráv v protokolu MQTT}
    \label{table:02:mqtt-subscribes}
\end{table}
