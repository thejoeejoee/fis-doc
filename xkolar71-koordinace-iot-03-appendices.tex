\chapter{Instalace firmwaru na uzel ESP32}\label{ch:instalator}

V~repozitáři \cli{fis-esp-firmware} se kromě samotného firmwaru pro uzly nachází i instalátor --
ten je založen na předpisu konfigurace pro utilitu \texttt{make} v~podobě souboru \cli{Makefile}.

Před samotnou instalací je nutné vytvořit konfigurační soubor pro uzly \cli{config.json}, který byl popsán v~kapitole
\ref{sec:detaily-z-implementace-jadra} -- jako vzor pro jeho vytvoření slouží šablonový soubor
\cli{config.template.json}, který obsahuje veškeré povinné klíče pro konfiguračního hodnoty nutné pro běh uzlu.

Kompletní instalace na čistý uzel probíhá pomocí \mbox{\cli{\$ make install PORT=/dev/ttyUSB0}}, kde
parametr \cli{PORT} definuje virtuální terminál, na kterém je zařízení připojeno -- uživatel musí disponovat právy na
práci s~tímto terminálem. Tento cíl postupně vykoná cíle \cli{erase-flash}, \cli{flash-micropython},
\cli{install-libs} a \cli{install-fis} -- v~případě selhální některé z~nich lze postupovat jednotlivě.

Jakmile je nainstalován firmware, lze pomocí cíle \cli{console} (či varianty s~resetem) otevřít konzoli na uzlu a
v~případě, že je vše v~pořádku, je v~lokálním prostoru jmen dostupná proměnná \cli{c}, jenž je referencí na firmware.
Manuálním voláním \cli{c.start()} lze firmware spustit a sledovat jeho logovací výstupy.

Pro automatické spouštění firmwaru při zapnutí je nutné volat \cli{\$ make enable-autoloader}, který na systém
souborů na uzlu umístí soubor \cli{main.py}, který má na starost automatické zavedení a reset v~případě chyby.
Analogicky poté \cli{\$ make disable-autoloader} vypne automatický start.

Soubor \cli{Makefile} obsahuje celkově tyto cíle (řazeno abecedně):
\begin{description}
    \item[\texttt{console}] Otevřen REPL\footnote{\uv{Read Execute Print Loop} (REPL) je postup použitý pro
    interaktivní vykonávání kódu interpretovaného jazyka v~konzoli.} konzoli interpretu jazyka na připojeném uzlu.
    \item[\texttt{console-with-reset}] Provede tvrdý reset připojeného uzlu a následně otevře konzoli.
    \item[\texttt{disable-autoloader}] Vypne automatický start firmwaru při zapnutí uzlu.
    \item[\texttt{enable-autoloader}] Zapne automatický start firmwaru při zapnutí uzlu.
    \item[\texttt{erase-flash}] Smaže kompletně pamět flash na připojeném uzlu.
    \item[\texttt{flash-micropython}] Nahraje stažený binární obraz interpretu jazyka MicroPython na připojený uzel.
    \item[\texttt{help}] Vypíše nápovědu k~cílům.
    \item[\texttt{install-fis}] Nahraje firmware společně se zaváděcím souborem \texttt{boot.py} na připojený uzel --
    kromě toho také vypne automatické zavádění firmwaru.
    \item[\texttt{install-libs}] Pomocí instalačního souboru provede vzdálené připojení uzlu k~internetu a následně
    nainstaluje potřebné závislosti na uzel.
    \item[\texttt{install}] Provede kompletní instalaci na uzel (smazání flash paměti, zavedení interpretu jazyka
    MicroPython, instalaci závislostí a samotného firmwaru).
    \item[\texttt{put-config}] Na připojený uzel nahraje konfigurační soubor \texttt{config.json} z~aktuálního adresáře.
    \item[\texttt{put-install}] Na uzel nahraje instalační skript.
    \item[\texttt{remote-deploy}] Na základě předaného identikátoru uzlu \texttt{NODE\_ID} provede vzdálené nasazení
    firmwaru skrz kanál MQTT mířící do konfigurační aplikace.
    \item[\texttt{remote-reset}] Na základě předaného identikátoru uzlu \texttt{NODE\_ID} provede vzdálený tvrdý
    reset uzlu skrz konfigurační aplikaci.
    \item[\texttt{reset-chip}] Provede tvrdý reset na připojeném uzlu.
\end{description}
