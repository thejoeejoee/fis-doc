%==============================================================================
% tento soubor pouzijte jako zaklad
% this file should be used as a base for the thesis
% Autoři / Authors: 2008 Michal Bidlo, 2018 Jaroslav Dytrych
% Kontakt pro dotazy a připomínky: dytrych@fit.vutbr.cz
% Contact for questions and comments: dytrych@fit.vutbr.cz
%==============================================================================
% kodovani: UTF-8 (zmena prikazem iconv, recode nebo cstocs)
% encoding: UTF-8 (you can change it by command iconv, recode or cstocs)
%------------------------------------------------------------------------------
% zpracování / processing: make, make pdf, make clean
%==============================================================================
% Soubory, které je nutné upravit: / Files which have to be edited:
%   xkolar71-koordinace-iot-20-literatura-bibliography.bib - literatura / bibliography
%   xkolar71-koordinace-iot-01-kapitoly-chapters.tex - obsah práce / the thesis content
%   xkolar71-koordinace-iot-30-prilohy-appendices.tex - přílohy / appendices
%==============================================================================
% \documentclass[]{fitthesis} % bez zadání - pro začátek práce, aby nebyl problém s překladem
% \documentclass[english]{fitthesis} % without assignment - for the work start to avoid compilation problem
\documentclass[zadani]{fitthesis} % odevzdani do wisu a/nebo tisk s barevnými odkazy - odkazy jsou barevné
%\documentclass[english,zadani]{fitthesis} % for submission to the IS FIT and/or print with color links - links are color
%\documentclass[zadani,print]{fitthesis} % pro černobílý tisk - odkazy jsou černé
%\documentclass[english,zadani,print]{fitthesis} % for the black and white print - links are black
%\documentclass[zadani,cprint]{fitthesis} % pro barevný tisk - odkazy jsou černé, znak VUT barevný
%\documentclass[english,zadani,cprint]{fitthesis} % for the print - links are black, logo is color
% * Je-li práce psaná v anglickém jazyce, je zapotřebí u třídy použít 
%   parametr english následovně:
%   If thesis is written in english, it is necessary to use 
%   parameter english as follows:
%      \documentclass[english]{fitthesis}
% * Je-li práce psaná ve slovenském jazyce, je zapotřebí u třídy použít 
%   parametr slovak následovně:
%   If the work is written in the Slovak language, it is necessary 
%   to use parameter slovak as follows:
%      \documentclass[slovak]{fitthesis}
% * Je-li práce psaná v anglickém jazyce se slovenským abstraktem apod., 
%   je zapotřebí u třídy použít parametry english a enslovak následovně:
%   If the work is written in English with the Slovak abstract, etc., 
%   it is necessary to use parameters english and enslovak as follows:
%      \documentclass[english,enslovak]{fitthesis}

% Základní balíčky jsou dole v souboru šablony fitthesis.cls
% Basic packages are at the bottom of template file fitthesis.cls
% zde můžeme vložit vlastní balíčky / you can place own packages here

% Kompilace po částech (rychlejší, ale v náhledu nemusí být vše aktuální)
% Compilation piecewise (faster, but not all parts in preview will be up-to-date)
\usepackage{subfiles}


% CUSTOM JOE
\newcolumntype{b}{X}
\newcolumntype{s}{>{\hsize=.5\hsize}X}
\newcolumntype{r}{>{\hsize=.5\hsize}X}
\usepackage{nameref}
\usepackage{subfig}
\usepackage[export]{adjustbox}
\usepackage{pifont}% http://ctan.org/pkg/pifont
\newcommand{\truemark}{\ding{51}}%
\newcommand{\falsemark}{\ding{55}}%
% END CUSTOM JOE

% Nastavení cesty k obrázkům
% Setting of a path to the pictures
\graphicspath{{figs/}{../figs/}}
%\graphicspath{{obrazky-figures/}{../obrazky-figures/}}

%---rm---------------
\renewcommand{\rmdefault}{lmr}%zavede Latin Modern Roman jako rm / set Latin Modern Roman as rm
%---sf---------------
\renewcommand{\sfdefault}{qhv}%zavede TeX Gyre Heros jako sf
%---tt------------
\renewcommand{\ttdefault}{lmtt}% zavede Latin Modern tt jako tt

% vypne funkci šablony, která automaticky nahrazuje uvozovky,
% aby nebyly prováděny nevhodné náhrady v popisech API apod.
% disables function of the template which replaces quotation marks
% to avoid unnecessary replacements in the API descriptions etc.
\csdoublequotesoff

% =======================================================================
% balíček "hyperref" vytváří klikací odkazy v pdf, pokud tedy použijeme pdflatex
% problém je, že balíček hyperref musí být uveden jako poslední, takže nemůže
% být v šabloně
% "hyperref" package create clickable links in pdf if you are using pdflatex.
% Problem is that this package have to be introduced as the last one so it 
% can not be placed in the template file.
\ifWis
\ifx\pdfoutput\undefined % nejedeme pod pdflatexem / we are not using pdflatex
\else
\usepackage{color}
\usepackage[unicode,colorlinks,hyperindex,plainpages=false,pdftex]{hyperref}
\definecolor{hrcolor-ref}{RGB}{223,52,30}
\definecolor{hrcolor-cite}{HTML}{2F8F00}
\definecolor{hrcolor-urls}{HTML}{092EAB}
\hypersetup{
linkcolor=hrcolor-ref,
citecolor=hrcolor-cite,
filecolor=magenta,
urlcolor=hrcolor-urls
}
\def\pdfBorderAttrs{/Border [0 0 0] }  % bez okrajů kolem odkazů / without margins around links
\pdfcompresslevel=9
\fi
\else % pro tisk budou odkazy, na které se dá klikat, černé / for the print clickable links will be black
\ifx\pdfoutput\undefined % nejedeme pod pdflatexem / we are not using pdflatex
\else
\usepackage{color}
\usepackage[unicode,colorlinks,hyperindex,plainpages=false,pdftex,urlcolor=black,linkcolor=black,citecolor=black]{hyperref}
\definecolor{links}{rgb}{0,0,0}
\definecolor{anchors}{rgb}{0,0,0}
\def\AnchorColor{anchors}
\def\LinkColor{links}
\def\pdfBorderAttrs{/Border [0 0 0] } % bez okrajů kolem odkazů / without margins around links
\pdfcompresslevel=9
\fi
\fi
% Řešení problému, kdy klikací odkazy na obrázky vedou za obrázek
% This solves the problems with links which leads after the picture
\usepackage[all]{hypcap}

% Informace o práci/projektu / Information about the thesis
%---------------------------------------------------------------------------
\projectinfo{
%Prace / Thesis
project={BP},            %typ práce BP/SP/DP/DR  / thesis type (SP = term project)
year={2019},             % rok odevzdání / year of submission
date=\today,             % datum odevzdání / submission date
%Nazev prace / thesis title
title.cs={Koordinace IoT na bázi MicroPythonu pomocí Node-RED},  % název práce v češtině či slovenštině (dle zadání) / thesis title in czech language (according to assignment)
title.en={Coordination of MiroPython-Based IoT by Means of Node-RED}, % název práce v angličtině / thesis title in english
title.length={13.5cm}, % nastavení délky bloku s titulkem pro úpravu zalomení řádku (lze definovat zde nebo níže) / setting the length of a block with a thesis title for adjusting a line break (can be defined here or below)
%Autor / Author
author.name={Josef},   % jméno autora / author name
author.surname={Kolář},   % příjmení autora / author surname
%author.title.p={Bc.}, % titul před jménem (nepovinné) / title before the name (optional)
%author.title.a={Ph.D.}, % titul za jménem (nepovinné) / title after the name (optional)
%Ustav / Department
department={UITS}, % doplňte příslušnou zkratku dle ústavu na zadání: UPSY/UIFS/UITS/UPGM / fill in appropriate abbreviation of the department according to assignment: UPSY/UIFS/UITS/UPGM
% Školitel / supervisor
supervisor.name={Vladimír},   % jméno školitele / supervisor name
supervisor.surname={Janoušek},   % příjmení školitele / supervisor surname
supervisor.title.p={doc. Ing.},   %titul před jménem (nepovinné) / title before the name (optional)
supervisor.title.a={Ph.D.},    %titul za jménem (nepovinné) / title after the name (optional)
% Klíčová slova / keywords
keywords.cs={Sem budou zapsána jednotlivá klíčová slova v českém (slovenském) jazyce, oddělená čárkami.}, % klíčová slova v českém či slovenském jazyce / keywords in czech or slovak language
keywords.en={Sem budou zapsána jednotlivá klíčová slova v anglickém jazyce, oddělená čárkami.}, % klíčová slova v anglickém jazyce / keywords in english
%keywords.en={Here, individual keywords separated by commas will be written in English.},
% Abstrakt / Abstract
abstract.cs={Do tohoto odstavce bude zapsán výtah (abstrakt) práce v českém (slovenském) jazyce.}, % abstrakt v českém či slovenském jazyce / abstract in czech or slovak language
abstract.en={Do tohoto odstavce bude zapsán výtah (abstrakt) práce v anglickém jazyce.}, % abstrakt v anglickém jazyce / abstract in english
%abstract.en={An abstract of the work in English will be written in this paragraph.},
% Prohlášení (u anglicky psané práce anglicky, u slovensky psané práce slovensky) / Declaration (for thesis in english should be in english)
declaration={Prohlašuji, že jsem tuto bakalářskou práci vypracoval samostatně pod vedením doc. Ing. Vladimíra Janouška, Ph.D.
% Další informace mi poskytli...
Uvedl jsem všechny literární prameny a publikace, ze kterých jsem čerpal.},
%declaration={Hereby I declare that this bachelor's thesis was prepared as an original author’s work under the supervision of Mr. X
% The supplementary information was provided by Mr. Y
% All the relevant information sources, which were used during preparation of this thesis, are properly cited and included in the list of references.},
% Poděkování (nepovinné, nejlépe v jazyce práce) / Acknowledgement (optional, ideally in the language of the thesis)
acknowledgment={V této sekci je možno uvést poděkování vedoucímu práce a těm, kteří poskytli odbornou pomoc
(externí zadavatel, konzultant, apod.).},
%acknowledgment={Here it is possible to express thanks to the supervisor and to the people which provided professional help
%(external submitter, consultant, etc.).},
% Rozšířený abstrakt (cca 3 normostrany) - lze definovat zde nebo níže / Extended abstract (approximately 3 standard pages) - can be defined here or below
%extendedabstract={Do tohoto odstavce bude zapsán rozšířený výtah (abstrakt) práce v českém (slovenském) jazyce.},
%faculty={FIT}, % FIT/FEKT/FSI/FA/FCH/FP/FAST/FAVU/USI/DEF
faculty.cs={Fakulta informačních technologií}, % Fakulta v češtině - pro využití této položky výše zvolte fakultu DEF / Faculty in Czech - for use of this entry select DEF above
faculty.en={Faculty of Information Technology}, % Fakulta v angličtině - pro využití této položky výše zvolte fakultu DEF / Faculty in English - for use of this entry select DEF above
% department.cs={Ústav matematiky}, % Ústav v češtině - pro využití této položky výše zvolte ústav DEF nebo jej zakomentujte / Department in Czech - for use of this entry select DEF above or comment it out
% department.en={Institute of Mathematics} % Ústav v angličtině - pro využití této položky výše zvolte ústav DEF nebo jej zakomentujte / Department in English - for use of this entry select DEF above or comment it out
}

% Rozšířený abstrakt (cca 3 normostrany) - lze definovat zde nebo výše / Extended abstract (approximately 3 standard pages) - can be defined here or above
%\extendedabstract{Do tohoto odstavce bude zapsán výtah (abstrakt) práce v českém (slovenském) jazyce.}

% nastavení délky bloku s titulkem pro úpravu zalomení řádku - lze definovat zde nebo výše / setting the length of a block with a thesis title for adjusting a line break - can be defined here or above
%\titlelength{14.5cm}


% řeší první/poslední řádek odstavce na předchozí/následující stránce
% solves first/last row of the paragraph on the previous/next page
\clubpenalty=10000
\widowpenalty=10000

% checklist
\newlist{checklist}{itemize}{1}
\setlist[checklist]{label=$\square$}

\begin{document}
    % Vysazeni titulnich stran / Typesetting of the title pages
    % ----------------------------------------------
    \maketitle
    % Obsah
    % ----------------------------------------------
    \setlength{\parskip}{0pt}

    {\setcounter{tocdepth}{1}\hypersetup{hidelinks}\tableofcontents}

    \listoftodos[Tohle všechno bys měl dodělat]

    % Seznam obrazku a tabulek (pokud prace obsahuje velke mnozstvi obrazku, tak se to hodi)
    % List of figures and list of tables (if the thesis contains a lot of pictures, it is good)
    \ifczech
    \renewcommand\listfigurename{Seznam obrázků}
    \fi
    \ifslovak
    \renewcommand\listfigurename{Zoznam obrázkov}
    \fi
    % \listoffigures

    \ifczech
    \renewcommand\listtablename{Seznam tabulek}
    \fi
    \ifslovak
    \renewcommand\listtablename{Zoznam tabuliek}
    \fi
    % \listoftables

    \ifODSAZ
    \setlength{\parskip}{0.5\bigskipamount}
    \else
    \setlength{\parskip}{0pt}
    \fi

    % vynechani stranky v oboustrannem rezimu
    % Skip the page in the two-sided mode
    \iftwoside
    \cleardoublepage
    \fi

    % Text prace / Thesis text
    % ----------------------------------------------
    \subfile{chapters/01-introduction.tex}
    \subfile{chapters/02-iot-principles.tex}
    \subfile{chapters/03-node-red.tex}
    \subfile{chapters/04-protocol.tex}
    \subfile{chapters/05-extensions.tex}
    \subfile{chapters/06-firmware.tex}
    \subfile{chapters/07-usage.tex}
    \subfile{chapters/08-conclusion.tex}


    % Kompilace po částech (viz výše, nutno odkomentovat)
    % Compilation piecewise (see above, it is necessary to uncomment it)
    %\subfile{projekt-01-uvod-introduction}
    % ...
    %\subfile{chapters/projekt-05-conclusion}


    % Pouzita literatura / Bibliography
    % ----------------------------------------------
    \ifslovak
    \makeatletter
    \def\@openbib@code{\addcontentsline{toc}{chapter}{Literatúra}}
    \makeatother
    \bibliographystyle{bib-styles/slovakiso}
    \else
    \ifczech
    \makeatletter
    \def\@openbib@code{\addcontentsline{toc}{chapter}{Literatura}}
    \makeatother
    \bibliographystyle{bib-styles/czechiso}
    \else
    \makeatletter
    \def\@openbib@code{\addcontentsline{toc}{chapter}{Bibliography}}
    \makeatother
    \bibliographystyle{bib-styles/englishiso}
    %  \bibliographystyle{alpha}
    \fi
    \fi
    \begin{flushleft}
        \bibliography{xkolar71-koordinace-iot-02-bibliography}
    \end{flushleft}

    % vynechani stranky v oboustrannem rezimu
    % Skip the page in the two-sided mode
    \iftwoside
    \cleardoublepage
    \fi

    % Prilohy / Appendices
    % ---------------------------------------------
    \appendix
    \ifczech
    \renewcommand{\appendixpagename}{Přílohy}
    \renewcommand{\appendixtocname}{Přílohy}
    \renewcommand{\appendixname}{Příloha}
    \fi
    \ifslovak
    \renewcommand{\appendixpagename}{Prílohy}
    \renewcommand{\appendixtocname}{Prílohy}
    \renewcommand{\appendixname}{Príloha}
    \fi
    %  \appendixpage

    % vynechani stranky v oboustrannem rezimu
    % Skip the page in the two-sided mode
    %\iftwoside
    %  \cleardoublepage
    %\fi

    \ifslovak
    %  \section*{Zoznam príloh}
    %  \addcontentsline{toc}{section}{Zoznam príloh}
    \else
    \ifczech
    %    \section*{Seznam příloh}
    %    \addcontentsline{toc}{section}{Seznam příloh}
    \else
    %    \section*{List of Appendices}
    %    \addcontentsline{toc}{section}{List of Appendices}
    \fi
    \fi
    \startcontents[chapters]
    \setlength{\parskip}{0pt}
    % seznam příloh / list of appendices
    % \printcontents[chapters]{l}{0}{\setcounter{tocdepth}{2}}

    \ifODSAZ
    \setlength{\parskip}{0.5\bigskipamount}
    \else
    \setlength{\parskip}{0pt}
    \fi

    % vynechani stranky v oboustrannem rezimu
    \iftwoside
    \cleardoublepage
    \fi

    % Přílohy / Appendices
    \chapter{Instalace firmwaru na uzel ESP32}\label{ch:instalator}

V~repozitáři \cli{fis-esp-firmware} se kromě samotného firmwaru pro uzly nachází i instalátor --
ten je založen na předpisu konfigurace pro utilitu \texttt{make} v~podobě souboru \cli{Makefile}.

Před samotnou instalací je nutné vytvořit konfigurační soubor pro uzly \cli{config.json}, který byl popsán v~kapitole
\ref{sec:detaily-z-implementace-jadra} -- jako vzor pro jeho vytvoření slouží šablonový soubor
\cli{config.template.json}, který obsahuje veškeré povinné klíče pro konfiguračního hodnoty nutné pro běh uzlu.

Kompletní instalace na čistý uzel probíhá pomocí \mbox{\cli{\$ make install PORT=/dev/ttyUSB0}}, kde
parametr \cli{PORT} definuje virtuální terminál, na kterém je zařízení připojeno -- uživatel musí disponovat právy na
práci s~tímto terminálem. Tento cíl postupně vykoná cíle \cli{erase-flash}, \cli{flash-micropython},
\cli{install-libs} a \cli{install-fis} -- v~případě selhální některé z~nich lze postupovat jednotlivě.

Jakmile je nainstalován firmware, lze pomocí cíle \cli{console} (či varianty s~resetem) otevřít konzoli na uzlu a
v~případě, že je vše v~pořádku, je v~lokálním prostoru jmen dostupná proměnná \cli{c}, jenž je referencí na firmware.
Manuálním voláním \cli{c.start()} lze firmware spustit a sledovat jeho logovací výstupy.

Pro automatické spouštění firmwaru při zapnutí je nutné volat \cli{\$ make enable-autoloader}, který na systém
souborů na uzlu umístí soubor \cli{main.py}, který má na starost automatické zavedení a reset v~případě chyby.
Analogicky poté \cli{\$ make disable-autoloader} vypne automatický start.

Soubor \cli{Makefile} obsahuje celkově tyto cíle (řazeno abecedně):
\begin{description}
    \item[\texttt{console}] Otevřen REPL\footnote{\uv{Read Execute Print Loop} (REPL) je postup použitý pro
    interaktivní vykonávání kódu interpretovaného jazyka v~konzoli.} konzoli interpretu jazyka na připojeném uzlu.
    \item[\texttt{console-with-reset}] Provede tvrdý reset připojeného uzlu a následně otevře konzoli.
    \item[\texttt{disable-autoloader}] Vypne automatický start firmwaru při zapnutí uzlu.
    \item[\texttt{enable-autoloader}] Zapne automatický start firmwaru při zapnutí uzlu.
    \item[\texttt{erase-flash}] Smaže kompletně pamět flash na připojeném uzlu.
    \item[\texttt{flash-micropython}] Nahraje stažený binární obraz interpretu jazyka MicroPython na připojený uzel.
    \item[\texttt{help}] Vypíše nápovědu k~cílům.
    \item[\texttt{install-fis}] Nahraje firmware společně se zaváděcím souborem \texttt{boot.py} na připojený uzel --
    kromě toho také vypne automatické zavádění firmwaru.
    \item[\texttt{install-libs}] Pomocí instalačního souboru provede vzdálené připojení uzlu k~internetu a následně
    nainstaluje potřebné závislosti na uzel.
    \item[\texttt{install}] Provede kompletní instalaci na uzel (smazání flash paměti, zavedení interpretu jazyka
    MicroPython, instalaci závislostí a samotného firmwaru).
    \item[\texttt{put-config}] Na připojený uzel nahraje konfigurační soubor \texttt{config.json} z~aktuálního adresáře.
    \item[\texttt{put-install}] Na uzel nahraje instalační skript.
    \item[\texttt{remote-deploy}] Na základě předaného identikátoru uzlu \texttt{NODE\_ID} provede vzdálené nasazení
    firmwaru skrz kanál MQTT mířící do konfigurační aplikace.
    \item[\texttt{remote-reset}] Na základě předaného identikátoru uzlu \texttt{NODE\_ID} provede vzdálený tvrdý
    reset uzlu skrz konfigurační aplikaci.
    \item[\texttt{reset-chip}] Provede tvrdý reset na připojeném uzlu.
\end{description}


    % Kompilace po částech (viz výše, nutno odkomentovat)
    % Compilation piecewise (see above, it is necessary to uncomment it)
    %\subfile{xkolar71-koordinace-iot-30-prilohy-appendices}

\end{document}
